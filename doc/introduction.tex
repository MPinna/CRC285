%-------------------------------------------------------------------------------
% File: introduction.tex
%
% Author: Marco Pinna
%         Created on 08/04/2022
%-------------------------------------------------------------------------------
\chapter{Introduction}\label{ch:introduction}
The work is organized as follows:
\begin{itemize}
	\item this chapter contains the description of the algorithm, some of its applications and possible architecture for the implementation
	\item in chapter \ref{ch:architecture} an in-depth explanation of the chosen architecture(s) is given
	\item in chapter \ref{ch:vhdl_code} the VHDL code of each component is shown and commented
	\item chapter \ref{ch:testplan} concerns the test plan and the testbenches used for the verification of the system
	\item chapter \ref{ch:synth_results} shows the results of the automated logic synthesis on the Xilinx FPGA Zync platform
	\item finally in chapter \ref{ch:conclusions} conclusions are drawn and some consideration about possible optimizations or different implementations are made
\end{itemize}

\section{Algorithm description}\label{sec:alg_description}
A Cyclic Redundancy Check (CRC) algorithm is a technique used in digital networks that uses redundant bits to detect accidental changes in digital data.\\
Let us supposed to have a \textit{sender} (S) and a \textit{receiver} (R) at the two end of a communication channel.\\
To each message M of \texttt{m} bits being sent by S, an additional section (the Frame Control Sequence, or FCS) of \texttt{f} redundant bits is added, whose value is computed performing a polynomial division between the message M (the dividend) and a polynomial G (the divisor) called \textit{generator} and calculating the remainder of such division.\\
Upon reception, R performs the same division and, depending on the value of the remainder, it is able to check whether the message M has been corrupted during transmission.\\

\noindent More in detail:
\begin{itemize}[leftmargin=0pt, topsep=0pt,itemsep=-1ex,partopsep=1ex,parsep=1ex]
	\item[-] let $M$ be an \texttt{m} bits long binary string.\\
	An \texttt{m-1} degree polynomial $M(x)$ is associated to it, such that the \textit{i}-th coefficient of the polynomial is equal to the \textit{i}-th bit of the string (e.g. $100101111 \Rightarrow x^{8} + x^{5} + x^{3} + x^{2} + x + 1$).\\
	\item[-] Let $G(x)$ be the generator polynomial whose binary representation is \texttt{f + 1} bits long (the degree of G(x) will therefore be \texttt{f}).\\
M is shifted to the left by \texttt{f} positions, padding to the right with \texttt{f} zeros. This corresponds to multiplying $M(x)$ by $x^f$.\\
	\item[-] The FCS is built as follows:\\
	a polynomial \textit{long division} between $x^{f}{\cdot}M(x)$ and $G(x)$ is performed, using finite field arithmetic on the Galois Field GF(2). Let $Q(x)$ and $R(X)$ be the quotient and the remainder of such division, respectively. \\
	It follows that 
	\begin{equation}
		x^{f}{\cdot}M(x) = Q(x)\cdot G(x) + R(x)
		\label{eq:polynomial1}
	\end{equation}
\\
	If we subtract $R(x)$ from both sides of the equation, we get
	\begin{equation}
			x^{f}{\cdot}M(x) - R(x) = Q(x)\cdot G(x)
			\label{eq:polynomial2}
	\end{equation}
	\item[-] The polynomial on the left-hand side of the equation is divisible by $G(x)$ and contains the original message M in the \texttt{m} highest bits and the FCS in the \texttt{f} lower bits (by construction, the degree of $R(X)$ is strictly less than \texttt{f}, so it can be represented on \texttt{f }bits).\\
	\item[-] This \texttt{m+f} bits long string M{$|$}FCS is what will be sent to the receiver.\\
	\item[-] The receiver can check the integrity of the message by dividing M{$|$}FCS by $G(x)$ and checking whether the remainder is equal to 0: if it is, then the message M has likely not been corrupted during transmission, otherwise it has and has to be discarded.
\end{itemize}
\hfill \break
In this particular implementation, M will be 56 bits long, the generator will be

	\begin{equation}
		\begin{split}
			x^{8} + x^{4} + x^{3} + x^{2} + 1
	  	\end{split}
	\quad\leftrightarrow\quad
  		\begin{split}
			100011101
  		\end{split}
	\label{eq:generator}
	\end{equation}
therefore the FCS will be 8 bits long, for a total of 64 bits to be sent for each message.
\paragraph{Example}
\mbox{}\\
Let us use as an example the transmission of the word ``hi'' encoded in ASCII.\\
The binary string corresponding to ``hi'' is:
	\begin{equation}
	01101000\:\:01101001
	\label{eq:example_binary}	
	\end{equation}
We first pad it with zeros
	\begin{equation}
	01101000\:\:01101001\:\:00000000
	\label{eq:example_padded}	
	\end{equation}
then we compute the FCS by performing the long division between \ref{eq:example_padded} and \ref{eq:generator}.\\

\newdimen\digitwidth
\settowidth\digitwidth{0}
\def~{\hspace{\digitwidth}}


\def\divrule#1#2{%
\noalign{\moveright#1\digitwidth%
\vbox{\hrule width#2\digitwidth}}}
\begin{tabular}[b]{@{}r@{}}
\begin{tabular}[t]{@{}l@{}}
011010000110100100000000\\
~100011101\\
\divrule{1}{9}
~0101111001\\
~~100011101\\
\divrule{2}{9}
~~00110010001\\
~~~~100011101\\
\divrule{4}{9}
~~~~0100011000\\
~~~~~100011101\\
\divrule{5}{9}
~~~~~000000101010000\\
~~~~~~~~~~~100011101\\
\divrule{11}{9}
~~~~~~~~~~~00100110100\\
~~~~~~~~~~~~~100011101\\
\divrule{13}{9}
~~~~~~~~~~~~~000\textbf{10100100}\\
\end{tabular}
\end{tabular}
\\
\hfill \break
The remainder R is $10100100$. R is subtracted from (\ref{eq:example_padded}) to create the FCS in the 8 lower bits, yielding the following string

	\begin{equation}
	01101000\:\:01101001\:\:10100100 .
	\label{eq:example_with_FCS}	
	\end{equation}


\section{Possible applications}\label{sec:possible_applications}
CRC algorithms are used extensively in various data transmission technologies and protocols: SATA, USB, Ethernet, CDMA, Bluetooth, etc.\\
They are also found in several standards, programs and file formats such as MPEG-2, PNG, Gzip, to name a few.\\
There are different implementations of CRC algorithms, each with its own polynomial. The choice of the polynomial (both its degree and its coefficients) depend on several factors such as the maximum desired overhead and the sensitivity to different errors.\\

\noindent An important remark to be made is that such algorithms protect against \textit{accidental} corruption of data, but they are not suited against \textit{intentional} alteration of data, since a valid CRC to attach to the tampered message can easily be computed once the algorithm is known.

\section{Possible architectures}\label{sec:possible_architectures}
Two different architectures were identified to implement the algorithm:
\begin{itemize}
\item the first one simply implements the long division shown in \ref{sec:alg_description} using shift registers, accumulators, counters , XOR gates, etc.\\
In this implementation the algorithm advances bit by bit, therefore in order to ``consume" the whole message (56 bits), a total of at least 56 clock cycles is needed.
\item the second one exploits the facts that the divisor (i.e. the generator) is fixed and that in GF(2) there is no carry, therefore adjacent bytes have no influence on each other during the algorithm. It is then possible to pre-compute the division for each possible byte and store the results in a LUT.\\
This allows to perform the division byte by byte, thus speeding up the whole process. This solution will be more efficient in terms of speed, at the expense of a greater use of resources on the FPGA since a 256x1B LUT has to be created.
\end{itemize}